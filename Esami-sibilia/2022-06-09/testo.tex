\documentclass{article}
\usepackage[utf8]{inputenc}
\usepackage{amsmath}
\begin{document}
00/06/22

\subsection{N 1}
Due corpi di massa $ m_1 $ (kg) ed $ m_2 $ (kg) sono collegati per mezzo di un filo inestensibile.
Il corpo di massa $ m_1 $ giace su di un tavolo orizzontale liscio,
mentre l'altro è poggiato su di un piano inclinato liscio, con un angolo di inclinazione $ \theta $.
Se nonostante la massa $ m_2 $ i due corpi rimangono fermi,
si determini il coefficiente di attrito statico tra la superficie del tavolo e la 
massa $ m_1 $, sapendo che questa inizia a muoversi se si applica una forza pari a 1.2 kg.

\subsection{N 2}
Una massa $ m = 1 {kg} $  viene lanciata verso il basso
con una velocità $ v_0 = 3 \text{m/s} $, da una quota $ h_0 = 1  \text{m} $
rispetto all’estremità libera di una molla di costante elastica $ k = 500 \text{N/m} $, 
posta lungo la verticale e con un estremo a terra. La massa comprime la molla fino ad una quota minima e viene poi rilanciata verso l’alto dalla molla stessa. 
Si calcoli la deformazione massima della molla e l’altezza massima raggiunta dalla massa.

\subsection{N 3}
Una mole di gas perfetto monoatomico, inizialmente a temperatura ambiente (T = 300 \, \text{K}), esegue una espansione isobara
irreversibile fino a raddoppiare il volume occupato. Determinare la variazione di energia interna del gas.

\subsection{N 4}
Nel circuito in fifura $C_1 = C_4 = 4 \mu F$  e $C_2 = C_3 = 1.33 \mu F$ e $E = 100 V$. Calcolare la capacità
equivalente e l'energia elettrostatica totale U. 
Il disegno è tagliato ma c'è il generatore sulla sbarretta vertyicale a sinistra (col positivo verso
l'alto), attaccato a $C_1$ e $C_2$  in serie in verticale, e attaccato a $C_3$  e $C_4$  in serie in verticale a sinistra.
Quindi $C_1$  e $C_2$ sono parallaeli a $C_3$ e $C_4$. Vabbè


\subsection{N 5}
Una spira quadrata conduttrice di lato $a= 10 cm$ giace sul piano xy. Un lato della spira è parallelo all'asse y 
e sitrova alla distanza $a$ dall'asse stesso. Un filo conduttore rettilineo infinito coincide con l'asse y.
Lubngo il filo scorre la corrente $I(t) = I_0 \sin\left(2 \pi f t \right)$, di ampiezza $I_0 = 10A$ e frequenza
$10 kHz$. Calcolare il valore massimo della forza elettromotrice indotta nella spira.
\end{document}

